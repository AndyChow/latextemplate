\documentclass{beamer}
\mode<presentation>

% 居中标题
\usetheme{Singapore} %定义了布局
%\usecolortheme{seahorse} %定义布局的外部填充颜色,比如标题和导航的颜色
%\usecolortheme{rose}     %定义布局的内部填充颜色,比如box和一些组件颜色

% 去掉导航
\setbeamertemplate{headline}{}

% 标题字体设置,所有的标题,包含frame,box,以及其他组件等
%\usefonttheme[onlylarge]{structuresmallcapsserif}
% 导航字体设置
%\usefonttheme[onlysmall]{structurebold}

% 仅设置标题页标题的字体,\titlepage里面可以看到效果
\setbeamerfont{title}{shape=\itshape,family=\rmfamily}
% 设置titlepage的字体前色和背色
\setbeamercolor{title}{fg=red!80!black,bg=red!20!white}

% 定义一个新的环境,类似于section,frame,定义方式类似与宏
\newenvironment{slide}[1]%1是参数
{\begin{frame}[fragile,environment=slide]
    \frametitle{#1}}%#1是引用参数
  {\end{frame}}



%-------------------------------------------------------------------------------
% Title
%-------------------------------------------------------------------------------
\title{There Is No Largest Prime Number}
\author[Euclid]{Euclid of Alexandria \\ \texttt{euclid@alexandria.edu}}
\date[ISPN ’80]{27th International Symposium of Prime Numbers}

%-------------------------------------------------------------------------------
% Document
%-------------------------------------------------------------------------------
\begin{document}
\section{Motivation}
\subsection{The Basic Problem That We Studied}

\begin{frame}
  \titlepage
\end{frame}

\begin{frame}[t]
  frame t vertically
\end{frame}

\begin{frame}[plain]
  \begin{centering}%
    %空白,标题和脚注都消失
  \end{centering}%
\end{frame}

\begin{frame}[shrink=5] %文字自动调节大小,这个特性要知道
Some evil endless slide that is 5\% too large.
\end{frame}

\begin{frame}{Frame Title}{subtitle}
  frame c center default
\end{frame}

\begin{frame}
  \frametitle{Outline}
  \tableofcontents[pausesections]
\end{frame}

\begin{frame}
  \frametitle{What Are Prime Numbers?}
  \begin{definition}
    A \alert{prime number} is a number that has exactly two divisors.
  \end{definition}
  \begin{example}
    \begin{itemize}
    \item 2 is prime (two divisors: 1 and 2).
      \pause
    \item 3 is prime (two divisors: 1 and 3).
      \pause
    \item 4 is not prime (\alert{three} divisors: 1, 2, and 4).
    \end{itemize}
  \end{example}
\end{frame}

\begin{frame}[t]
  \frametitle{There Is No Largest Prime Number}
  \framesubtitle{The proof uses \textit{reductio ad absurdum}.}
  \begin{theorem}
    There is no largest prime number.
  \end{theorem}
  \begin{proof}
    \begin{enumerate}
      %1-表明从第一个frame开始显示,2-3表明在2-3里面出现,-3表明前3个frame出现,可以指定范围
      %The specification <1-> means “from slide 1 on.”
    \item<1-> Suppose $p$ were the largest prime number.
    \item<2-> Let $q$ be the product of the first $p$ numbers.
    \item<3-> Then $q + 1$ is not divisible by any of them.
    \item<1-> But $q + 1$ is greater than $1$, thus divisible by some prime
      number not in the first $p$ numbers.\qedhere 
      %\qedhere is used to put the qed symbol at the end of the line inside the enumeration
    \end{enumerate}
  \end{proof}
%  \uncover<4->{The proof used \textit{reductio ad absurdum}.}
% \only command simply “throws its argument away” and the argument does not
% occupy any space. This leads to different heights of the text on the first three slides and on the fourth slide.
  \only<4->{The proof used \textit{reductio ad absurdum}.}
\end{frame}

\begin{frame}
  \frametitle{What’s Still To Do?}
  \begin{columns}[t] %vertically align the block
    \column{.5\textwidth}
    \begin{block}{Answered Questions}
      How many primes are there?
    \end{block}
    \column{.5\textwidth}
    \begin{block}{Open Questions}
      Is every even number the sum of two primes?
      $^{\cite{Goldbach1742}}$
    \end{block}
  \end{columns}
\end{frame}

\begin{frame}[fragile] %Verbatim text 按照原字体格式显示
  \frametitle{An Algorithm For Finding Primes Numbers.}
  % verbatim环境中所有latex的命令都失去意义,被视为普通的文本
  % 如果要使用命令,需要使用semiverbatim,半文本的意思
\begin{verbatim} 
int main (void)
{
    std::vector<bool> is_prime (100, true);
    for (int i = 2; i < 100; i++)
        if (is_prime[i])
        {
            std::cout << i << " ";
            for (int j = i; j < 100; is_prime [j] = false, j+=i);
        }
    return 0;
}
\end{verbatim}
  \begin{uncoverenv}<2>
  Note the use of \verb|std::|.
\end{uncoverenv}
\end{frame}

\begin{thebibliography}{10}
\bibitem{Goldbach1742}[Goldbach, 1742]
  Christian Goldbach.
  \newblock A problem we should try to solve before the ISPN ’43 deadline,
  \newblock \emph{Letter to Leonhard Euler}, 1742.
\end{thebibliography}

\end{document}
