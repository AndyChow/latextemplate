%%%%%%%%%%%%%%%%%%%%%%%%%%%%%%%%%%%%%%%%%
% Compact Laboratory Book
% LaTeX Template
% Version 1.0 (4/6/12)
%
% This template has been downloaded from:
% http://www.LaTeXTemplates.com
%
% Original author:
% Joan Queralt Gil (http://phobos.xtec.cat/jqueralt) using the labbook class by
% Frank Kuster (http://www.ctan.org/tex-archive/macros/latex/contrib/labbook/)
%
% License:
% CC BY-NC-SA 3.0 (http://creativecommons.org/licenses/by-nc-sa/3.0/)
%
% Important note:
% This template requires the labbook.cls file to be in the same directory as the
% .tex file. The labbook.cls file provides the necessary structure to create the
% lab book.
%
% The \lipsum[#] commands throughout this template generate dummy text
% to fill the template out. These commands should all be removed when 
% writing lab book content.
%
% HOW TO USE THIS TEMPLATE 
% Each day in the lab consists of three main things:
%
% 1. LABDAY: The first thing to put is the \labday{} command with a date in 
% curly brackets, this will make a new section showing that you are working
% on a new day.
%
% 2. EXPERIMENT/SUBEXPERIMENT: Next you need to specify what 
% experiment(s) and subexperiment(s) you are working on with a 
% \experiment{} and \subexperiment{} commands with the experiment 
% shorthand in the curly brackets. The experiment shorthand is defined in the 
% 'DEFINITION OF EXPERIMENTS' section below, this means you can 
% say \experiment{pcr} and the actual text written to the PDF will be what 
% you set the 'pcr' experiment to be. If the experiment is a one off, you can 
% just write it in the bracket without creating a shorthand. Note: if you don't 
% want to have an experiment, just leave this out and it won't be printed.
%
% 3. CONTENT: Following the experiment is the content, i.e. what progress 
% you made on the experiment that day.
%
%%%%%%%%%%%%%%%%%%%%%%%%%%%%%%%%%%%%%%%%%

%----------------------------------------------------------------------------------------
%	PACKAGES AND OTHER DOCUMENT CONFIGURATIONS
%----------------------------------------------------------------------------------------                               

\documentclass[fontsize=11pt, % Document font size
                             paper=a4, % Document paper type
                             oneside, % Shifts odd pages to the left for easier reading when printed, can be changed to oneside
                             captions=tableheading,
                             index=totoc,
                             hyperref]{labbook}
 
\usepackage[bottom=10em]{geometry} % Reduces the whitespace at the bottom of the page so more text can fit

\usepackage[english]{babel} % English language
\usepackage{lipsum} % Used for inserting dummy 'Lorem ipsum' text into the template

%\usepackage[utf8]{inputenc} % Uses the utf8 input encoding
\usepackage[T1]{fontenc} % Use 8-bit encoding that has 256 glyphs

\usepackage[osf]{mathpazo} % Palatino as the main font
\linespread{1.05}\selectfont % Palatino needs some extra spacing, here 5% extra
\usepackage[scaled=.88]{beramono} % Bera-Monospace
\usepackage[scaled=.86]{berasans} % Bera Sans-Serif

\usepackage{booktabs,array} % Packages for tables

\usepackage{amsmath} % For typesetting math
\usepackage{graphicx} % Required for including images
\usepackage{etoolbox}
\usepackage[norule]{footmisc} % Removes the horizontal rule from footnotes
\usepackage{lastpage} % Counts the number of pages of the document

\usepackage[dvipsnames]{xcolor}  % Allows the definition of hex colors
\definecolor{titleblue}{rgb}{0.16,0.24,0.64} % Custom color for the title on the title page
\definecolor{linkcolor}{rgb}{0,0,0.42} % Custom color for links - dark blue at the moment

\addtokomafont{title}{\Huge\color{titleblue}} % Titles in custom blue color
\addtokomafont{chapter}{\color{OliveGreen}} % Lab dates in olive green
\addtokomafont{section}{\color{Sepia}} % Sections in sepia
\addtokomafont{pagehead}{\normalfont\sffamily\color{gray}} % Header text in gray and sans serif
\addtokomafont{caption}{\footnotesize\itshape} % Small italic font size for captions
\addtokomafont{captionlabel}{\upshape\bfseries} % Bold for caption labels
\addtokomafont{descriptionlabel}{\rmfamily}
\setcapwidth[c]{10cm} % Right align caption text c is "center", l is "left", r is "right"
\setkomafont{footnote}{\sffamily} % Footnotes in sans serif

\deffootnote[4cm]{4cm}{1em}{\textsuperscript{\thefootnotemark}} % Indent footnotes to line up with text

\DeclareFixedFont{\textcap}{T1}{phv}{bx}{n}{1.5cm} % Font for main title: Helvetica 1.5 cm
\DeclareFixedFont{\textaut}{OT1}{phv}{bx}{n}{0.8cm} % Font for author name: Helvetica 0.8 cm

\usepackage[nouppercase,headsepline]{scrpage2} % Provides headers and footers configuration
\pagestyle{scrheadings} % Print the headers and footers on all pages
\clearscrheadfoot % Clean old definitions if they exist

\automark[chapter]{chapter}
\ohead{\headmark} % Prints outer header

\setlength{\headheight}{25pt} % Makes the header take up a bit of extra space for aesthetics
\setheadsepline{.4pt} % Creates a thin rule under the header
\addtokomafont{headsepline}{\color{lightgray}} % Colors the rule under the header light gray

\ofoot[\normalfont\normalcolor{\thepage\ |\  \pageref{LastPage}}]{\normalfont\normalcolor{\thepage\ |\  \pageref{LastPage}}} % Creates an outer footer of: "current page | total pages"

% These lines make it so each new lab day directly follows the previous one i.e. does not start on a new page - comment them out to separate lab days on new pages
\makeatletter
\patchcmd{\addchap}{\if@openright\cleardoublepage\else\clearpage\fi}{\par}{}{}
\makeatother
\renewcommand*{\chapterpagestyle}{scrheadings}

% These lines make it so every figure and equation in the document is numbered consecutively rather than restarting at 1 for each lab day - comment them out to remove this behavior
\usepackage{chngcntr}
\counterwithout{figure}{labday}
\counterwithout{equation}{labday}

% Hyperlink configuration
\usepackage[
    pdfauthor={}, % Your name for the author field in the PDF
    pdftitle={Laboratory Journal}, % PDF title
    pdfsubject={}, % PDF subject
    bookmarksopen=true,
    linktocpage=true,
    urlcolor=linkcolor, % Color of URLs
    citecolor=linkcolor, % Color of citations
    linkcolor=linkcolor, % Color of links to other pages/figures
    backref=page,
    pdfpagelabels=true,
    plainpages=false,
    colorlinks=true, % Turn off all coloring by changing this to false
    bookmarks=true,
    pdfview=FitB]{hyperref}

\usepackage[stretch=10]{microtype} % Slightly tweak font spacing for aesthetics

%\setlength\parindent{0pt} % Uncomment to remove all indentation from paragraphs

\usepackage{zhfontcfg}% Chinese fonts
\usepackage{indentfirst} % indention for Chinese
\setlength{\parindent}{2em}

\usepackage{algorithm}
\usepackage{algorithmicx}
\usepackage{algpseudocode}

\usepackage{listings}
\lstset{
%  numbers=left,
  keywordstyle=\color{blue},
  commentstyle=\color[cmyk]{1,0,1,0},
  frame=single,
  breaklines,
  extendedchars=false,
  tabsize=4,
  showspaces=false
}
\usepackage{xcolor}

%source code
\usepackage{verbments}
%设定中文等宽字体,否则的话代码中中文显示乱码
\setmonofont{AR PL UKai CN}
%----------------------------------------------------------------------------------------
%	DEFINITION OF EXPERIMENTS
%----------------------------------------------------------------------------------------

% Template: \newexperiment{<abbrev>}[<short form>]{<long form>}
% <abbrev> is the reference to use later in the .tex file in \experiment{}, the <short form> is only used in the table of contents and running title - it is optional, <long form> is what is printed in the lab book itself

\newexperiment{学习情况}[学习情况]{学习情况}

\newexperiment{napi}[NAPI技术]{NAPI技术}
\newsubexperiment{napi_driver_design}[NAPI驱动设计]{NAPI驱动设计}
\newsubexperiment{para_qs_OpenMP}[快速排序OpenMP实现]{快速排序OpenMP实现}
\newsubexperiment{para_qs_result}[实验结果与分析]{实验结果与分析}

%----------------------------------------------------------------------------------------

\begin{document}

%----------------------------------------------------------------------------------------
%	TITLE PAGE
%----------------------------------------------------------------------------------------

\title{\erhao\hei{学习情况} \\[1cm]  
\textaut{2014-4-22}}

\author{
    %\textaut{周永强 Yongqiang Zhou}\\ \\ % Your name
	\erhao{\hei{周永强}} \\[1cm]
	\fzqt{六院八队} % Your degree
}
\date{} % No date by default, add \today if you wish to include the publication date

\maketitle % Title page

\printindex
\tableofcontents % Table of contents

\newpage % Start lab look on a new page

\begin{addmargin}[0cm]{0cm} % Makes the text width much shorter for a compact look

\pagestyle{scrheadings} % Begin using headers

%----------------------------------------------------------------------------------------
%	LAB BOOK CONTENTS
%----------------------------------------------------------------------------------------
\labday{说明}
\experiment{学习情况}
\indent 课程已经进入中期,有许多课程任务要做,主要是前期拖延积攒下来的任务。实验室那边去的事情不多,所以这段时间去的比较少,于是便以课程为主。我选的课普遍几乎都是应用性比较强的课程,Linux内核要做源码分析,高级并行程序要做并行程序设计,分布式系统要结合分布式做一个android应用,路由器原理又要做netmagic平台上的SDN计数器。前三个任务已经完成过半,最后一个还没有开始做,因为我的硬件基础几乎等于0,所以比较吃力。\\
\indent 后面的内容是有关于内核网卡驱动中napi技术的介绍和分析,也是为选的课程作业,结合了网上别人的分析和自己阅读源代码后写出来的,由于表达能力不是太好,写得可能不是太明白,同时限于水平问题,可能会有错的地方。
%NAPI

\labday{NAPI}
\experiment{NAPI简介}
\indent \en{NAPI("New API")}是对设备驱动程序报文处理的一个框架,其设计的初衷是为了提高高速网络设备的性能。\en{NAPI}通过两种机制来提高性能,分别是:中断减负和报文节流。\\
\indent 中断减负:典型的报文处理流程是当报文到达网卡时,设备产生中断通知\en{CPU},然后\en{CPU}响应中断,从网卡读取报文数据交由上层协议栈。但是对于高速网卡设备来说,单位时间内会产生数以千计的中断通知系统响应报文处理,频繁的中断会严重影响系统的性能。\en{NAPI}允许驱动在大流量的网络环境时禁用中断,从而降低中断次数,降低系统因中断造成的负担。\\
\indent 报文节流:当大量报文涌向系统时,系统必然会产生丢包,相比交由系统处理时才丢弃报文,能够尽早地丢弃报文更好。使用\en{NAPI}的驱动能够在报文到达网络适配器时就将其丢弃,内核根本无法看到它们,这里丢弃报文指的是指因系统处理不过而丢弃的报文。\\
\indent 新的网卡在开发驱动时应该使用\en{NAPI}技术以获得更好的性能。

\experiment{NAPI驱动设计}
\indent \en{NAPI}的设计思想就一句话,综合了中断和轮询。当网络数据到达网卡时,产生中断,此时中断程序响应,但是中断程序并不处理数据,而是禁用中断,并且通过\en{NAPI}的\en{poll}函数以轮询的方式对数据进行处理,而且在适当的时候重新启用中断,具体什么时候,后面还会讲。使用\en{NAPI}共分为四步骤,分别为:定义,注册与初始化,调度和删除。\\
\indent 下面以e1000驱动为例子,详细说明使用\en{NAPI}技术的驱动程序接收数据的全过程。\\

\experiment{NAPI定义}
\indent \en{NAPI}使用结构体\en{napi\_struct}描述,如果驱动程序要使用该技术的话,可以在自己的结构体中内嵌实现。在e1000驱动adapter的定义中可以看到napi\_struct的存在。\\
%\begin{lstlisting}[language=C]
\begin{pyglist}[language=c,caption={e1000.h},listingname=\textbf{Program},
	listingnamefont=\sffamily\bfseries\color{yellow},%
        captionfont=\sffamily\color{white},captionbgcolor=gray,
        fvset={frame=bottomline,framerule=4pt,rulecolor=\color{gray}}
        ]
/*e1000.h*/
struct e1000_adapter {
  /*...*/
  /*嵌入napi结构*/
  struct napi_struct napi;
  /*...*/
}
\end{pyglist}

%\end{lstlisting}

\experiment{NAPI注册与初始化}
\indent napi\_struct在定义以后,必须初始化和注册才能够使用。其初始化与注册是在e1000\_probe函数中完成,该函数是探测函数,会在。。(PCI routine)。。执行。注册函数为: \\ netif\_napi\_add(dev, \&napi, poll, weight) \\前两个参数是设备和napi结构指针,重要的是第三个参数,第三个参数是函数指针,指向使用napi时要使用的轮询函数,在注册的时候必须指明该参数,然后当使用轮询接收数据时,会回调该函数。在e1000驱动中,轮询函数为e1000\_clean。
%\begin{lstlisting}[language=C]
\begin{pyglist}[language=c,caption={e1000\_probe},listingname=\textbf{Program},
	listingnamefont=\sffamily\bfseries\color{yellow},%
        captionfont=\sffamily\color{white},captionbgcolor=gray,
        fvset={frame=bottomline,framerule=4pt,rulecolor=\color{gray}}
        ]
/*e1000_main.c*/
static int e1000_probe(struct pci_dev *pdev, const struct pci_device_id *ent)
{
  struct net_device *netdev;
  struct e1000_adapter *adapter;
  /*...*/
  SET_NETDEV_DEV(netdev, &pdev->dev);
  pci_set_drvdata(pdev, netdev);
  adapter = netdev_priv(netdev);
  adapter->netdev = netdev;
  adapter->pdev = pdev;
  /*...*/
  //初始化网卡驱动的相关操作,包括open,close等
  netdev->netdev_ops = &e1000_netdev_ops;
  e1000_set_ethtool_ops(netdev);
  /*...*/
  //napi初始化与注册,这里面要指明轮询函数为e1000_clean
  netif_napi_add(netdev, &adapter->napi, e1000_clean, 64);
  /*...*/
}
\end{pyglist}
%\end{lstlisting}

\experiment{调度NAPI}
\indent 在讲napi的调度之前必须先说明一下中断,中断在e1000\_open中注册完成,注册完成后,每当网卡有数据到达,便会产生中断,然后调用中断处理程序,在中断处理程序中禁用中断,并且调度napi轮询接收数据。napi的调度函数为:\\
void napi\_schedule(struct napi\_struct *napi); \\
或:\\
if (napi\_schedule\_prep(napi))\\
    \_\_napi\_schedule(napi);\\
二者的效果一样,只不过后者是先测试当前已经在进行napi调度了。\\

%\begin{lstlisting}[language=C]
\begin{pyglist}[language=c,caption={e1000\_open},listingname=\textbf{Program},
	listingnamefont=\sffamily\bfseries\color{yellow},%
        captionfont=\sffamily\color{white},captionbgcolor=gray,
        fvset={frame=bottomline,framerule=4pt,rulecolor=\color{gray}}
        ]
/*e1000_open*/
static int e1000_open(struct net_device *netdev)
{
    struct e1000_adapter *adapter = netdev_priv(netdev);
    struct e1000_hw *hw = &adapter->hw;
    /*...*/
    //网卡设置,其中apapter->clean_rx 在其中进行设置
    e1000_configure(adapter);
    /*...*/
    //申请中断
    err = e1000_request_irq(adapter);
    if (err)
        goto err_req_irq;
    /*...*/
    napi_enable(&adapter->napi);
    e1000_irq_enable(adapter);
    /*...*/
}

//中断注册
static int e1000_request_irq(struct e1000_adapter *adapter)
{
    struct net_device *netdev = adapter->netdev;
    //中断处理函数设定
    irq_handler_t handler = e1000_intr;
    int irq_flags = IRQF_SHARED;
    int err;
    //申请中断,中断处理函数为e1000_intr
    err = request_irq(adapter->pdev->irq, handler, irq_flags, netdev->name, netdev);
    /*...*/
}

static irqreturn_t e1000_intr(int irq, void *data)
{
    /*...*/
    /* disable interrupts, without the synchronize_irq bit */
    ew32(IMC, ~0);
    E1000_WRITE_FLUSH();
    //调度napi进行数据接收
    if (likely(napi_schedule_prep(&adapter->napi))) {
        adapter->total_tx_bytes = 0;
        adapter->total_tx_packets = 0;
        adapter->total_rx_bytes = 0;
        adapter->total_rx_packets = 0;
        __napi_schedule(&adapter->napi);
    }
    /*...*/
}
\end{pyglist}
%\end{lstlisting}

\indent 当napi调度以后,后面的工作便是轮询接收数据,轮询函数已经在e1000\_probe中设定为e1000\_clean。\\
%\begin{lstlisting}[language=C]
\begin{pyglist}[language=c,caption={e1000\_clean},listingname=\textbf{Program},
	listingnamefont=\sffamily\bfseries\color{yellow},%
        captionfont=\sffamily\color{white},captionbgcolor=gray,
        fvset={frame=bottomline,framerule=4pt,rulecolor=\color{gray}}
        ]
static int e1000_clean(struct napi_struct *napi, int budget)
{
    struct e1000_adapter *adapter = container_of(napi, struct e1000_adapter, napi);
    int tx_clean_complete = 0, work_done = 0;
    //发送数据报文,这是发送流程里面的,暂时可以不关心
    tx_clean_complete = e1000_clean_tx_irq(adapter, &adapter->tx_ring[0]);
    //接收数据,这里clean_rx 为函数指针,指向数据接收函数 
    //budget为接收数据量的限制
    adapter->clean_rx(adapter, &adapter->rx_ring[0], &work_done, budget);

    if (!tx_clean_complete)
        work_done = budget;

    /* If budget not fully consumed, exit the polling mode */
    //如果budget没有用完,也即接收的数据量没有超多限度,
    //那么说明轮询可以暂时结束了,重新启用中断,响应后续到来的数据
    if (work_done < budget) {
        if (likely(adapter->itr_setting & 3))
            e1000_set_itr(adapter);
        napi_complete(napi);
        if (!test_bit(__E1000_DOWN, &adapter->flags))
            //重新启用中断
            e1000_irq_enable(adapter);
    }
    return work_done;
}
\end{pyglist}
%\end{lstlisting}

\indent 最后还要看一下clean\_rx函数,这个函数之所以设定成指针是为了灵活使用接收函数,不同型号的网卡可能不一样,但是原理都是一样的,该指针的初始化在e1000\_open函数中的e1000\_configure()中进行设定,而该函数又调用e1000\_configure\_rx进行设定接收函数,假定接收函数被设为e1000\_clean\_rx\_irq(设成其他函数也一样,原理是相同的),接收函数完成功能见代码。\\
%\begin{lstlisting}[language=C]
\begin{pyglist}[language=c,caption={e1000\_clean\_rx\_irq},listingname=\textbf{Program},
	listingnamefont=\sffamily\bfseries\color{yellow},%
        captionfont=\sffamily\color{white},captionbgcolor=gray,
        fvset={frame=bottomline,framerule=4pt,rulecolor=\color{gray}}
        ]

static bool e1000_clean_rx_irq(struct e1000_adapter *adapter,
			       struct e1000_rx_ring *rx_ring,
			       int *work_done, int work_to_do)
{
  /*...*/
  i = rx_ring->next_to_clean;
  rx_desc = E1000_RX_DESC(*rx_ring, i);
  buffer_info = &rx_ring->buffer_info[i];

  while (rx_desc->status & E1000_RXD_STAT_DD) {
    //sk_buff代表报文,下面接收完毕就上交协议栈做下一步的处理
    //数据也就从网卡驱动进入内核协议栈进行解析了
    struct sk_buff *skb;
    u8 status;

    if (*work_done >= work_to_do)
      break;
    (*work_done)++;
   /* read descriptor and rx_buffer_info after status DD */
    rmb(); 

    skb = buffer_info->skb;
    buffer_info->skb = NULL;

    prefetch(skb->data - NET_IP_ALIGN);
    /*...*/

    length = le16_to_cpu(rx_desc->length);
    /*...*/
    process_skb:
      total_rx_bytes += (length - 4); /* don't count FCS */
      total_rx_packets++;

    /*...*/
    skb_put(skb, length);
    /*...*/
    e1000_receive_skb(adapter, status, rx_desc->special, skb);

    next_desc:
      rx_desc->status = 0;
    /*...*/
    }
}
\end{pyglist}
%\end{lstlisting}
\experiment{删除NAPI}
\indent 消注册在网卡消注册的时候进行调用,调用函数为netif\_napi\_del(),具体的代码不再分析。

%-----------------------------------------



%-----------------------------------------


\end{addmargin}

%----------------------------------------------------------------------------------------
%	BIBLIOGRAPHY
%----------------------------------------------------------------------------------------

\begin{thebibliography}{9}

\bibitem{lamport94}
\emph{http://www.linuxfoundation.org/node/add/wiki?gids[]=5066}.

\bibitem{kernel}
\emph{Linux kernel 2.6.34}.

\end{thebibliography}

%----------------------------------------------------------------------------------------

\end{document}
