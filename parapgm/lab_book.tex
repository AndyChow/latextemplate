%%%%%%%%%%%%%%%%%%%%%%%%%%%%%%%%%%%%%%%%%
% Compact Laboratory Book
% LaTeX Template
% Version 1.0 (4/6/12)
%
% This template has been downloaded from:
% http://www.LaTeXTemplates.com
%
% Original author:
% Joan Queralt Gil (http://phobos.xtec.cat/jqueralt) using the labbook class by
% Frank Kuster (http://www.ctan.org/tex-archive/macros/latex/contrib/labbook/)
%
% License:
% CC BY-NC-SA 3.0 (http://creativecommons.org/licenses/by-nc-sa/3.0/)
%
% Important note:
% This template requires the labbook.cls file to be in the same directory as the
% .tex file. The labbook.cls file provides the necessary structure to create the
% lab book.
%
% The \lipsum[#] commands throughout this template generate dummy text
% to fill the template out. These commands should all be removed when 
% writing lab book content.
%
% HOW TO USE THIS TEMPLATE 
% Each day in the lab consists of three main things:
%
% 1. LABDAY: The first thing to put is the \labday{} command with a date in 
% curly brackets, this will make a new section showing that you are working
% on a new day.
%
% 2. EXPERIMENT/SUBEXPERIMENT: Next you need to specify what 
% experiment(s) and subexperiment(s) you are working on with a 
% \experiment{} and \subexperiment{} commands with the experiment 
% shorthand in the curly brackets. The experiment shorthand is defined in the 
% 'DEFINITION OF EXPERIMENTS' section below, this means you can 
% say \experiment{pcr} and the actual text written to the PDF will be what 
% you set the 'pcr' experiment to be. If the experiment is a one off, you can 
% just write it in the bracket without creating a shorthand. Note: if you don't 
% want to have an experiment, just leave this out and it won't be printed.
%
% 3. CONTENT: Following the experiment is the content, i.e. what progress 
% you made on the experiment that day.
%
%%%%%%%%%%%%%%%%%%%%%%%%%%%%%%%%%%%%%%%%%

%----------------------------------------------------------------------------------------
%	PACKAGES AND OTHER DOCUMENT CONFIGURATIONS
%----------------------------------------------------------------------------------------                               

\documentclass[fontsize=11pt, % Document font size
                             paper=a4, % Document paper type
                             oneside, % Shifts odd pages to the left for easier reading when printed, can be changed to oneside
                             captions=tableheading,
                             index=totoc,
                             hyperref]{labbook}
 
\usepackage[bottom=10em]{geometry} % Reduces the whitespace at the bottom of the page so more text can fit

\usepackage[english]{babel} % English language
\usepackage{lipsum} % Used for inserting dummy 'Lorem ipsum' text into the template

%\usepackage[utf8]{inputenc} % Uses the utf8 input encoding
\usepackage[T1]{fontenc} % Use 8-bit encoding that has 256 glyphs

\usepackage[osf]{mathpazo} % Palatino as the main font
\linespread{1.05}\selectfont % Palatino needs some extra spacing, here 5% extra
\usepackage[scaled=.88]{beramono} % Bera-Monospace
\usepackage[scaled=.86]{berasans} % Bera Sans-Serif

\usepackage{booktabs,array} % Packages for tables

\usepackage{amsmath} % For typesetting math
\usepackage{graphicx} % Required for including images
\usepackage{etoolbox}
\usepackage[norule]{footmisc} % Removes the horizontal rule from footnotes
\usepackage{lastpage} % Counts the number of pages of the document

\usepackage[dvipsnames]{xcolor}  % Allows the definition of hex colors
\definecolor{titleblue}{rgb}{0.16,0.24,0.64} % Custom color for the title on the title page
\definecolor{linkcolor}{rgb}{0,0,0.42} % Custom color for links - dark blue at the moment

\addtokomafont{title}{\Huge\color{titleblue}} % Titles in custom blue color
\addtokomafont{chapter}{\color{OliveGreen}} % Lab dates in olive green
\addtokomafont{section}{\color{Sepia}} % Sections in sepia
\addtokomafont{pagehead}{\normalfont\sffamily\color{gray}} % Header text in gray and sans serif
\addtokomafont{caption}{\footnotesize\itshape} % Small italic font size for captions
\addtokomafont{captionlabel}{\upshape\bfseries} % Bold for caption labels
\addtokomafont{descriptionlabel}{\rmfamily}
\setcapwidth[c]{10cm} % Right align caption text c is "center", l is "left", r is "right"
\setkomafont{footnote}{\sffamily} % Footnotes in sans serif

\deffootnote[4cm]{4cm}{1em}{\textsuperscript{\thefootnotemark}} % Indent footnotes to line up with text

\DeclareFixedFont{\textcap}{T1}{phv}{bx}{n}{1.5cm} % Font for main title: Helvetica 1.5 cm
\DeclareFixedFont{\textaut}{OT1}{phv}{bx}{n}{0.8cm} % Font for author name: Helvetica 0.8 cm

\usepackage[nouppercase,headsepline]{scrpage2} % Provides headers and footers configuration
\pagestyle{scrheadings} % Print the headers and footers on all pages
\clearscrheadfoot % Clean old definitions if they exist

\automark[chapter]{chapter}
\ohead{\headmark} % Prints outer header

\setlength{\headheight}{25pt} % Makes the header take up a bit of extra space for aesthetics
\setheadsepline{.4pt} % Creates a thin rule under the header
\addtokomafont{headsepline}{\color{lightgray}} % Colors the rule under the header light gray

\ofoot[\normalfont\normalcolor{\thepage\ |\  \pageref{LastPage}}]{\normalfont\normalcolor{\thepage\ |\  \pageref{LastPage}}} % Creates an outer footer of: "current page | total pages"

% These lines make it so each new lab day directly follows the previous one i.e. does not start on a new page - comment them out to separate lab days on new pages
\makeatletter
\patchcmd{\addchap}{\if@openright\cleardoublepage\else\clearpage\fi}{\par}{}{}
\makeatother
\renewcommand*{\chapterpagestyle}{scrheadings}

% These lines make it so every figure and equation in the document is numbered consecutively rather than restarting at 1 for each lab day - comment them out to remove this behavior
\usepackage{chngcntr}
\counterwithout{figure}{labday}
\counterwithout{equation}{labday}

% Hyperlink configuration
\usepackage[
    pdfauthor={}, % Your name for the author field in the PDF
    pdftitle={Laboratory Journal}, % PDF title
    pdfsubject={}, % PDF subject
    bookmarksopen=true,
    linktocpage=true,
    urlcolor=linkcolor, % Color of URLs
    citecolor=linkcolor, % Color of citations
    linkcolor=linkcolor, % Color of links to other pages/figures
    backref=page,
    pdfpagelabels=true,
    plainpages=false,
    colorlinks=true, % Turn off all coloring by changing this to false
    bookmarks=true,
    pdfview=FitB]{hyperref}

\usepackage[stretch=10]{microtype} % Slightly tweak font spacing for aesthetics

%\setlength\parindent{0pt} % Uncomment to remove all indentation from paragraphs

\usepackage{zhfontcfg}% Chinese fonts
\usepackage{indentfirst} % indention for Chinese
\setlength{\parindent}{2em}

\usepackage{algorithm}
\usepackage{algorithmicx}
\usepackage{algpseudocode}

%----------------------------------------------------------------------------------------
%	DEFINITION OF EXPERIMENTS
%----------------------------------------------------------------------------------------

% Template: \newexperiment{<abbrev>}[<short form>]{<long form>}
% <abbrev> is the reference to use later in the .tex file in \experiment{}, the <short form> is only used in the table of contents and running title - it is optional, <long form> is what is printed in the lab book itself

\newexperiment{quicksort}[快速排序]{快速排序算法}

\newsubexperiment{subexp_example}[子实验例程1]{这是子实验例程1}
\newsubexperiment{subexp_example2}[子实验例程2]{这是子实验例程2}
\newsubexperiment{subexp_example3}[子实验例程3]{这是子实验例程3}

%----------------------------------------------------------------------------------------

\begin{document}

%----------------------------------------------------------------------------------------
%	TITLE PAGE
%----------------------------------------------------------------------------------------

\title{\textcap{高级并行程序设计实验} \\[1cm]  
\textaut{Beginning 02-04-2014}}

\author{
    %\textaut{周永强 Yongqiang Zhou}\\ \\ % Your name
	\erhao{\hei{周永强}} \\[1cm]
	\fzqt{六院八队} % Your degree
}
\date{} % No date by default, add \today if you wish to include the publication date

\maketitle % Title page

\printindex
\tableofcontents % Table of contents

\newpage % Start lab look on a new page

\begin{addmargin}[0cm]{0cm} % Makes the text width much shorter for a compact look

\pagestyle{scrheadings} % Begin using headers

%----------------------------------------------------------------------------------------
%	LAB BOOK CONTENTS
%----------------------------------------------------------------------------------------
%----------------------------------------------------------------------------------------
% 实验一:快速排序算法
%----------------------------------------------------------------------------------------
\labday{快速排序}
快速排序是一种基本的排序算法,其时间复杂度为$O(n\log{n})$,它的基本思
想是:对于给定的无序序列$R[1,n]$,从中选取一个元素作为“标兵”(一般选
取第一个元素或最后一个元素),以“标兵”为基准将序列划分为两个子序列$R[1,i-1]$
和$R[i,n]$,且左边的无序子区中记录的所有关键字均小于等于基准的关键字,右边的无序子区中记
录的所有关键字均大于等于基准的关键字,当子序列不空时,递归调用子上述过
程,程序运行结束时,序列有序。 \\
\indent 本节中先给出快速排序的串行化算法,然后对算法进行分析,引出并行的快速排
序算法,并且分别使用\en{MPI}和\en{OpenMP}实现并行的快速排序,最后对算
法的性能进行评估,给出评价和结果分析。
%评估主要根据加速比等性能指标,绘制图表

%-----------------------------------------

\experiment{quicksort}
%算法的串行伪代码,并行化分析并行代码
\begin{algorithm}
\caption{快速排序算法}\label{ser_quicksort}
\begin{algorithmic}[1]
\Function{\color{titleblue}{quicksort}}{$data, i, j$}\Comment{quicksort from i to j}
   \If{$i<j$}
       \State $r \gets$ partition($data, i, j$) 
   \EndIf
   \State quicksort($data, i, r-1$)
   \State quicksort($data, r+1, j$)
\EndFunction
\end{algorithmic}
\end{algorithm}

\begin{algorithm}
\caption{partition算法}\label{partition}
\begin{algorithmic}[1]
\Function{partition}{$data, k, l$}
   \State $pivo \gets data[l]$
   \State $i \gets k-l$ 
   \For {$j \gets k,l-1$}
       \If{$data[i] \le pivo$} %\le <= \ge is >=
           \State $i \gets i+1$
           \State exchange $data[i]$ and $data[j]$
       \EndIf
   \EndFor
   \State exchange $data[i+1]$ and $data[l]$
   \State return $i+1$
\EndFunction
\end{algorithmic}
\end{algorithm}

%-----------------------------------------

\subexperiment{subexp_example}
%分析如何用mpi进行实现

%-----------------------------------------

\subexperiment{subexp_example2}
%分析如何用openmp进行实现

%-----------------------------------------

\subexperiment{subexp_example3}
%结论

%----------------------------------------------------------------------------------------
% 实验二 Cannon方法的矩阵乘
%----------------------------------------------------------------------------------------
\labday{矩阵相乘的Cannon方法}
矩阵相互乘
\experiment{Cannon方法的矩阵乘}

\end{addmargin}

%----------------------------------------------------------------------------------------
%	BIBLIOGRAPHY
%----------------------------------------------------------------------------------------

\begin{thebibliography}{9}

\bibitem{lamport94}
Leslie Lamport,
\emph{\LaTeX: A Document Preparation System}.
Addison Wesley, Massachusetts,
2nd Edition,
1994.

\end{thebibliography}

%----------------------------------------------------------------------------------------

\end{document}
